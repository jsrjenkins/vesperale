%%% This is the Main tex document that performs the layout. 
%%% You will need to compile this using LuaLaTeX with the --shell-escape option

\documentclass[a4paper,twoside,12pt]{article}

%%% Preferences to change are here.
%%% Note that only the Sundays xiv-xv are supported for the moment
%%% Others will be added later.
\newcommand{\sunday}{xvii}%% CHANGE THIS TO WHICH SUNDAY AFTER PENTECOST
\newcommand{\lang}{polish} %which language translation. %
%%%% Note that the translation files are in the 'translatio' directory
%%% For the moment only polish is available besides english

%%%% some other preferences, not necessary to change.
\setlength\parindent{0cm} % Paragraph indentation
\usepackage[top=15mm, bottom=20mm, outer=17mm, inner=17mm]{geometry}% Margins
\usepackage[autocompile]{gregoriotex}% for layout of gregorian chant
\gresetlinecolor{gregoriocolor}%% red score lines in chants

%%%% some experimental preferences, not yet very effective:
%\setlength{\columnsep}{5mm}%% separation between columns
%\setlength{\columnseprule}{0.2pt}%% column rule

%%% For ordinary Sundays after Pentecost [for the moment, others will be added]

%%% You shouldn't need to make modifications after this point,
%%% unless you know what you are doing!

\usepackage{fontspec}% necessary for gregorio
\usepackage{libertine}% the font
\usepackage{xstring}% for calendar comparisons
\usepackage{paracol}% for parallel columns

\usepackage[latin,\lang]{babel} % the languages for hyphenation, etc.
\usepackage{vesperale} %% vespers style sheet has macros for typesetting
\usepackage{calendar} %% for the selection of propria

\newcommand{\SUNDAY}{\MakeUppercase{\sunday}}
\begin{document}
  \begin{center}%
    \huge{\textbf{Dominica in II Vesperas}}\\
    \large{\SUNDAY\ post Pentecosten}%%
  \end{center}
%%%%
\section*{Initium}%%
\versiculus{or--deus_in_adjutorium_simplex--solesmes}%
%%%%%
\section*{Psalmi}%% Psalms%
\subsection*{Psalmus 1}
\psalmus{an--dixit_dominus_domino--solesmes}{109-7c2}%
\subsection*{Psalmus 2}
\psalmus{an--magna_opera_domini--solesmes}{110-3b}%
\subsection*{Psalmus 3}
\psalmus{an--qui_timet_dominum--solesmes}{111-4g}%
\subsection*{Psalmus 4}
\psalmus{an--sit_nomen_domini_in--solesmes}{112-7c}%
\subsection*{Psalmus 5}
\psalmus{an--deus_autem_noster--solesmes}{113-per}%
\section*{Capitulum}%Chapter
\text{capitulum-ordinarium}
\section*{Hymnus}%Hymn
\hymnus{hy--lucis_creator_optime--solesmes}{dirigatur}{sicut}%
\section*{Canticum}%Magnificat
\canticum{pentecostes}{\sunday}
\section*{Oratio}%
\text{oratio-\sunday}%%
\section*{Conclusio}%
\versiculus{or--benedicamus_xi--solesmes}%%% dismissal
\vskip .5cm%%
\text{fidelium}%%%
%%end of document%
\end{document}%
%%% Emacs stuff %%%

%%% Local Variables:
%%% mode: latex
%%% TeX-master: t
%%% TeX-engine: luatex
%%% TeX-command-extra-options: "-shell-escape"
%%% End:

